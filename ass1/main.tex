\documentclass[a4paper]{article}

\usepackage{fullpage}
\usepackage{listings}
\usepackage[usenames,dvipsnames]{color}
\usepackage{hyperref}
\usepackage{amsmath}

\definecolor{light-gray}{gray}{0.95}
\lstset{tabsize=4,
	breaklines=true,
	breakatwhitespace=true,
	basicstyle=\scriptsize,
	language={Haskell},
	showstringspaces=false,
	numberstyle=\tiny,
	escapeinside={--(}{)--},
	numbers=left,
	backgroundcolor=\color{light-gray},
	frame=single,
}

\renewcommand{\labelenumi}{\textbf{(\alph{enumi})}}

\DeclareMathOperator*{\ord}{ord}

\title{Functional Programming lab 1}
\author{Ren\'e Mellema (s2348802) \and Xeryus Stokkel (s2332795)}

\begin{document}

\maketitle

\section*{Exercise 1}
\begin{enumerate}
	\item All the code for this exercise is at the bottom of this exercise.
	\item From $k=12$ onwards the processing time is noticeable, the processing time is then 0.26s compared to 0.08s for $k=11$. At $k=16$ we are just before the minute barrier (46.33s) while for $k=17$ the time it takes to compute has crossed the 3 minute barrier (190.41s). For $k=18$ it took 669.4s, at this point we decided to not run for $k=19,k=20$ as the computing time is clearly exponential and we have more excercises to do.
	\item Time for calculations is still about as poor as for this function as it does not need to do the \texttt{isPrime} test if the Fermat's theorem test has failed. This means that computation time can be cut down as \texttt{isPrime} is very expensive. The length of the list of pseudoprimes does grow exponentially with larger $k$ in the same way as it did with computing time in the previous question.
	\item This code is also available at the bottom of the exercise.
	\item The table with the number of 2-pseudoprimes on the interval \lstinline{[3..2^k]} for $2 \le k \le 22$ can be found in \autoref{tab:exe}
\end{enumerate}

\begin{table}[h]
	\centering
	\caption{Number of 2-pseudoprimes on the interval \lstinline{[3..2^k]}}
	\label{tbl:exe}
	\begin{tabular}{r|r}
		$k$  & \# psp2 \\
		\hline
		2  & 0   \\
		3  & 0   \\
		4  & 0   \\
		5  & 0   \\
		6  & 0   \\
		7  & 0   \\
		8  & 0   \\
		9  & 1   \\
		10 & 3   \\
		11 & 8   \\
		12 & 13  \\
		13 & 19  \\
		14 & 32  \\
		15 & 45  \\
		16 & 64  \\
		17 & 89  \\
		18 &     \\
		19 &     \\
		20 &     \\
		21 &     \\
		22 &     \\
	\end{tabular}
\end{table}

\lstinputlisting[caption=\lstname, label=code:naive]{naive.hs}

\section*{Exercise 2}
\lstinputlisting[caption=\lstname, label=code:smarter]{smarter.hs}

The proof that $q-1$ must be a multiple of $\ord_a(p)$ is as follows:

$a^{\ord_a(p)} - 1$ is a multiple of $p$, since $a^{\ord_a(p)}$ \textbf{mod} $p=1$. By the definition, it is also the smallest multiple of $p$. This means, that all other numbers $a^x$ for which $a^x$ \textbf{mod} $p=1$ holds must also be multiples of $a^{\ord_a(p)}$ since this is the first value that is one more than $p$. 

So in order for $a^{q-1}$ \textbf{mod} $p=1$ to hold, $a^{q-1}$ must be a multiple of $a^{\ord_a(p)}$. Since we are multiplying, we can leave out the $a$ and thuss follows that $q-1$ must be a multiple of $\ord_a(p)$. 

\begin{enumerate}
	\item Our \texttt{order} function is defined at line \autoref{func:order} in \autoref{code:smarter}. We tested both \texttt{order} (our function) and \texttt{order'} (the function given in the assignment) for all primes from 3 to $2^n$ for $10 \leq n \leq 16$  and noted the runtime in \autoref{tbl:order}. As we can see our implementation is between 2 and 10 times as fast as the naive implementation.
	
\begin{table}[h]
	\centering
	\caption{Runtimes of \texttt{order} vs. \texttt{order'} for all primes from 3 to $2^n$ where $10 \leq n \leq 16$}
	\label{tbl:order}
	\begin{tabular}{r|r|r}
		$n$ & \texttt{order} runtime (s) & \texttt{order'} runtime (s) \\
		\hline
		10 & 0.10 & 0.11 \\
		11 & 0.30 & 0.31 \\
		12 & 1.15 & 1.15 \\
		13 & 5.34 & 4.15 \\
		14 & 26.51 & 15.81 \\
		15 & 150.17 & 60.74 \\
		16 & 861.89 & 227.91
	\end{tabular}
\end{table}
	\item The function is implemented on line \autoref{func:oddPspTo} of \autoref{code:smarter}. This function generates a list \texttt{ns} which contains all numbers to test. It contains all numbers of $p \cdot (k \cdot \text{ord}_a(p) +1)$ where $p$ is from the set of prime numbers that is smaller than $2^13$ and $k$ is from $\{1, 2 \ldots (\frac{\text{upb}}{p \cdot \text{ord}_a(p)} \}$. These are all the composites that could be a pseudoprimes in the range from 2 to \texttt{upb}.
	\item See line \autoref{func:isPrime} in \autoref{code:smarter}. This simply tests whether the given $n$ is prime according to Fermat's little theorem and it also checks if it isn't in the list of known 2-pseudoprimes. This is enough to determine whether a number is prime. The list of known 2-pseudoprimes is cached so it needs to be calculated once and then it is very fast for subsequent calls.
	\item See \autoref{tbl:cntPrimes} for runtimes of the \texttt{cntPrimes} function as defined on line \autoref{func:cntPrimes} in \autoref{code:smarter}. All of these runtimes would be 781 seconds longer if \texttt{oddPspTO25} didn't cache the pseudoprimes to ($2^25$). If this had to be recalculated for every time cntPrimes would run then the program would be a lot more impractical to run. For these tests we pre-cached \texttt{oddPspTO25} so that none of the tests is affected by it having to precalculate this list.

\begin{table}[h]
	\centering
	\caption{Runtimes for \texttt{cntPrimes n} for $n=2^k$ where $2 \leq k \leq 25$}
	\label{tbl:cntPrimes}
	\begin{tabular}{r|r}
		$k$ & Runtime (s) \\
		\hline
		 2 & 0.01 \\
		 3 & 0.00 \\
		 4 & 0.00 \\
		 5 & 0.00 \\
		 6 & 0.00 \\
		 7 & 0.01 \\
		 8 & 0.01 \\
		 9 & 0.03 \\
		10 & 0.04 \\
		11 & 0.06 \\
		12 & 0.12 \\
		13 & 0.22 \\
		14 & 0.42 \\
		15 & 0.88 \\
		16 & 1.86 \\
		17 & 3.99 \\
		18 & 8.41 \\
		19 & 18.32 \\
		20 & 39.67 \\
		21 & 79.12 \\
		22 & 153.51 \\
		23 & 310.00 \\
		24 & 645.14 \\
		25 & 1341.68 \\
	\end{tabular}

\end{table}
\end{enumerate}

\end{document}